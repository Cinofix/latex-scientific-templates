\documentclass[12pt,a4paper,twoside,english]{report}

%\usepackage[a4paper,bindingoffset=0.2in,%
%top=0.9in,bottom=0.8in,%
%footskip=.35in]{geometry}


\usepackage[a4paper,top=0.9in,bottom=0.8in,
footskip=.35in]{geometry}

\usepackage{url,amsfonts,epsfig}
\usepackage{afterpage}
\usepackage{amsmath,amssymb}
\usepackage{rotating}
\usepackage[scriptsize]{caption2}
\usepackage{tikz} % for background logo in frontespizio
\usepackage{placeins}
\hyphenation{a-gen-tiz-za-zio-ne}
\setlength{\paperwidth}{22. cm} % old 21
\setlength{\paperheight}{29.7 cm}
\setlength{\oddsidemargin} {2. cm}
\setlength{\evensidemargin} {2. cm}
\addtolength{\oddsidemargin} {-0.4 cm}
\addtolength{\evensidemargin} {-0.4 cm}
\usepackage[english]{babel}
%\usepackage[latin1]{inputenc}
\usepackage[utf8]{inputenc}
\usepackage{hyperref}
\hypersetup{hidelinks=true}
\usepackage{listings}
\usepackage{algpseudocode,algorithm,algorithmicx}

\newcommand*\Let[2]{\State #1 $\gets$ #2}
\algrenewcommand\algorithmicrequire{\textbf{Input Data:}}
\algrenewcommand\algorithmicensure{\textbf{Output:}}
\usepackage{amsthm}
\usepackage{amsmath}
\usepackage{amsfonts}
\usepackage{amssymb}
\usepackage{mathtools}
\DeclarePairedDelimiter{\ceil}{\lceil}{\rceil}
\DeclarePairedDelimiter{\floor}{\lfloor}{\rfloor}
\usepackage{booktabs}

\usepackage[numbers]{natbib}

\usepackage{graphicx}
\usepackage{comment}
\usepackage{fancyhdr}
\usepackage{framed}
\usepackage{lastpage}
\usepackage{enumitem}
\usepackage{mdwlist}
\usepackage{placeins}
\usepackage{float}
\usepackage{courier}
\usepackage{tikz}

\usepackage{tikz}
%\renewcommand{\captionfont}{\normalfont \sffamily \itshape \small}
\renewcommand{\captionfont}{\normalfont \scriptsize}

\renewcommand{\contentsname}{Indice}
\newcommand{\hrefoot}[1]{\footnote{\href{#1}{#1}}}
\newcommand{\footprofile}[2]{\footnote{\href{#1}{#2}}}



%\def\qed{\ \ \vrule height6pt width5pt depth3pt}
\def\qed{\ \vrule height5pt width5pt depth0pt}
 
\newcommand{\pf}{{\sc Proof} }
\newcommand{\ep}{{\mbox{ }\nolinebreak{$\rule{2mm}{2mm}$}}}
\newcommand{\op}{\diamondsuit}
\newcommand{\cL}{{\cal L}}
\newcommand{\cN}{{\cal N}}
\newcommand{\cF}{{\cal F}}  
\newcommand{\QL}{Q_n [ {\cal L} ]}
\newcommand{\cNL}{{\cal N} [{\cal L} ]}
\newcommand{\diam}{{\rm diam}}
\newcommand{\pfo}{{\sc Proof} (outline) }
\newcommand{\OR}{\mbox{{\sc or}}}
%\newcommand{\TH}{\mbox{{\sc Th}}}
\newcommand{\iffe}{{\bf if}}
\newcommand{\then}{{\bf then}}
\newcommand{\els}{{\bf else}}
\newcommand{\beg}{{\bf begin}}
\newcommand{\e}{{\bf end}}   
\newcommand{\while}{{\bf while}}
\newcommand{\du}{{\bf do}}
\newcommand{\for}{{\bf for}}
\newcommand{\tu}{{\bf to}}   
\newcommand{\var}{{\bf var}} 
\newcommand{\of}{{\bf of}}   
\newcommand{\con}{{\bf with}}

\newcommand{\infig}[3]{\begin{figure*}[t]
\centerline{\epsfbox{#1}}
\caption{\label{#2} #3} \end{figure*}}


\newcommand{\image}[3]{ % 1 image 2 caption 3 size
	\begin{figure}[h!]
	  \centering
	  \includegraphics[width=#3\textwidth]{#1} 
	  \caption{#2}
	\end{figure}
	\FloatBarrier
}

\newcommand{\imageb}[2]{ % 1 image 2 size
	\begin{figure}[H]
		\centering
		\includegraphics[width=#2\textwidth]{#1} 
	\end{figure}
	\FloatBarrier
}

\newcommand{\imageLabel}[4]{ % 1 image 2 caption 3 size
	\begin{figure}[H]
		\centering
		\includegraphics[width=#3\textwidth]{#1} 
		\caption{#2}
		\label{fig:#4}
	\end{figure}
	\FloatBarrier
}
\newcommand{\Z}{\mathbb{Z}}

%Theorem definitions
\theoremstyle{plain}
\newtheorem{thm}{Theorem}[section] % reset theorem numbering for each chapter
\theoremstyle{definition}
\newtheorem{defn}[thm]{Definition} % definition numbers are dependent on theorem numbers
\newtheorem{exmp}[thm]{Example} % same for example numbers

\newcommand{\chaptercontent}{
	\section{Basics}
	\begin{defn}Here is a new definition.\end{defn}
	\begin{thm}Here is a new theorem.\end{thm}
	\begin{thm}Here is a new theorem.\end{thm}
	\begin{exmp}Here is a good example.\end{exmp}
	\subsection{Some tips}
	\begin{defn}Here is a new definition.\end{defn}
	\section{Advanced stuff}
	\begin{defn}Here is a new definition.\end{defn}
	\subsection{Warnings}
	\begin{defn}Here is a new definition.\end{defn}
}


\setlength\parindent{0pt} % No indent


\renewcommand{\vec}{\mathbf}
